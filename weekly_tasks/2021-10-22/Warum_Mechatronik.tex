\documentclass [11pt,letterpaper ,twoside ,openright ]{book}
\usepackage{helvet}
\renewcommand{\familydefault}{\sfdefault}
	
\usepackage{natbib} % Required to change bibliography style to APA
\bibliographystyle{plain}

\begin{document}

Computer, Auto, eigentlich alles um dich herum funktioniert automatisch. Aber wie kann das sein? Du willst wissen, wie all die Dinge funktionieren? Dann studiere Mechatronik! Im Laufe des Studiums werden sich mit der Zeit alle technischen Geräte von selbst erklären da du aus den drei technischen Teilbereichen Elektrotechnik, Informatik und Maschinenbau die Grundlagen, welche für die meisten technischen Anwendungen relevant sind, lernst. Dieses Wissen kann dir keiner mehr nehmen und du kannst weiterhin frei entscheiden, was du damit anfängst. Denn nach dem Mechatronikstudium stehen dir alle Türen offen. Du kannst dich auf nahezu jeden Fachbereich spezialisieren oder aber die Schnittstellen der Fachbereiche koordinieren. Lass dir nach dem Abi nicht deine Möglichkeiten nehmen und fokussiere dich nicht zu früh, denn wer weiß wofür du dich nach dem Studium interessierst. Ein Mechatronikstudium im Bachelor qualifiziert dich ebenso für ein jeden weitern technischen Zweig, als auch natürlich für eine Mechatronikmasterstudium. Egal, ob du gut in Mathe oder Physik bist, im Mechatronikstudium kannst du diese Dinge von Grund auf lernen und anwenden. Der Stoff fängt bei Null an.  



\end{document}