\chapter{Einleitung}
Die vorliegende Sammlung an Aufsätzen, im Folgenden Portfolio genannt, spiegelt die in dem Kurs "Wissenschaftliches Schreiben und Präsentieren für Mechatronik" erlernten Kenntnisse und Fähigkeiten des Verfassers wieder. Die Erstellung des Portfolios schulte den Verfasser mit wöchentlichen Abgaben im Erstellen wissenschaftlicher Texte und Arbeiten. \\\\%Relevanz
Die entstandenen Aufsätze sind in überarbeiteter Version in diesem Portfolio zusammengefasst und im Anhang sind die jeweiligen Rohversionen zu finden, teils mit markierten Verbesserungen, welche in Partnerarbeit entstanden sind. So lässt sich eine Veränderung der Fähigkeiten und ein Lernprozess nachvollziehen.\\\\
Dementsprechend ist das Erreichen eines Lernfortschritts das Ziel dieses Portfolios. Ob dieses Ziel erreicht wird kann nicht genau validiert werden, jedoch kann aufgrund der Rohversionen abschließend eine Einschätzung gegeben werden.\\\\
Das Schreiben von wissenschaftlichen Texten und die damit verbundene Auseinandersetzung mit fremden, sowie bekannten Themen ist nötig, um wissenschaftlich arbeiten zu können. Die Beleuchtung eines Themas von allen Seiten und die daraus folgenden Ableitungen und Schlussfolgerungen können allgemeingültige Fakten liefern. Erarbeitete Lösungsvorschläge können politischen EntscheidungsträgerInnen helfen, die richtigen Entscheidungen zu treffen. Um dieser Verantwortung gerecht zu werden und dementsprechend genau arbeiten zu können muss das wissenschaftliche Schreiben gelernt und geübt werden.\\\\
Die Inhalte der Texte befassen sich vorwiegend mit dem Klimawandel und der Frage, wie dieser mithilfe von unterschiedlichen Technologien und Aktionen auf ein bestimmtes Maß begrenzt werden kann. Dabei wird zum Beispiel über Elektromobilität und erneuerbare Energien gesprochen. Die Themen tragen zu aktuell in der Öffentlichkeit geführten Debatten bei und liefern Lösungsansätze für tiefgreifende Probleme.\\\\
Des weiteren gibt es Aufsätze zu dem Studiengang Mechatronik und zu im Körper eingepflanzten Chips, eine Artikelrezension mit dem Thema Bewegung eines Roboterarms, eine Prozessgrafik zu einem Kaufprozess und abschließend eine Projektskizze zu einer Bachelorarbeit.
