\chapter{Fazit/Reflexion}
In dem vorliegenden Portfolio wurde auf diverse Themen eingegangen, welche sich einerseits mit Technologien und andererseits mit dem Umgang des Menschen mit der Natur und dem Klimawandel beschäftigen. Dabei wurde zu Beginn die Frage gestellt, ob der menschengemachte Klimawandel noch begrenzt werden kann. Diese Frage wurde in den vorliegenden Aufsätzen nicht abschließend geklärt. Dies ist aufgrund der Komplexität des Themas aber auch nicht in diesem Umfang möglich oder wurde nicht angestrebt. Es wurden Lösungsansätze aufgezeigt und Fakten gesammelt, welche das Problem angehen und beziffern.\\\\
Es sei des Weiteren angemerkt, dass, wie in \ref{ha04} schon angemerkt, nicht nur der Energiesektor und die neuen Technologien eine Rolle spielen. Auch die Nahrungsmittelproduktion trägt mit knapp einem fünftel zu der Treibhausgasproduktion bei. Und gerade hier können die Aktionen eines jeden Einzelnen eine Rolle spielen. Mit jeder Entscheidung anstatt eines tierischen ein rein pflanzliches Produkt zu kaufen und zu essen, kann ein Einfluss auf den fortschreitenden Klimawandel genommen werden \cite{poore_reducing_2018}. Auch wenn die Hilflosigkeit über den geringen eigenen Einfluss auf große Probleme erdrückend ist, basiert die zukünftige Welt und Umwelt auf den eigenen zu jedem Zeitpunkt getroffenen Entscheidungen.
