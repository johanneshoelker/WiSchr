\chapter{Fazit/Reflexion}
Die Umsetzung des Ziels dieses Portfolios, einen Lernfortschritt in der Fähigkeit des wissenschaftlichen Schreibens zu erreichen, lässt sich in unterschiedlichen Aspekten erkennen. Allem voran die kausale Verkettung von Argumenten zu einer fundierten Begründung des betreffenden Themas und das einhergehende Aufzeigen von Problemen und Verbesserungsvorschlägen hat sich im Laufe des gesamten Kurses konstant verbessert. Die Begründung des eigenen Standpunktes wird damit deutlicher. Auf dem Gebiet der Grammatik war schon zu Beginn ein sicherer Umgang zu erkennen, jedoch wurden neue Fehler, besonders den erweiterten Infinitiv betreffend, nachträglich eingebaut und zeigen somit keine Verbesserung des Lernfortschritts. Der Einsatz von wissenschaftlichen Arbeiten dagegen wurde in den überarbeiteten Versionen weiter ausgebaut und das Lesen und Analysieren wissenschaftlicher Texte hat sich verbessert, was an dem vermehrten Einsatz von Quellen erkennbar ist. Aus persönlicher Einschätzung kann gesagt werden, dass der Umgang mit wissenschaftlichem Arbeiten im Allgemeinen während des gesamten Kurses sicherer geworden ist.\\\\
In dem vorliegenden Portfolio wurde auf diverse Themen eingegangen, welche sich einerseits mit Technologien und andererseits mit dem Umgang des Menschen mit der Natur und dem Klimawandel beschäftigen. Dabei wurde zu Beginn die Frage gestellt, ob der menschengemachte Klimawandel noch begrenzt werden kann. Diese Frage wurde in den vorliegenden Aufsätzen nicht abschließend geklärt. Dies ist aufgrund der Komplexität des Themas aber auch nicht in diesem Umfang möglich oder wurde nicht angestrebt. Es wurden Lösungsansätze aufgezeigt und Fakten gesammelt, welche das Problem angehen und beziffern.\\\\
Es sei des Weiteren angemerkt, dass, wie in 2.4 schon angemerkt, nicht nur der Energiesektor und die neuen Technologien eine Rolle spielen, um den fortschreitenden Klimawandel aufzuhalten. Auch die Nahrungsmittelproduktion trägt mit knapp einem fünftel zu der Treibhausgasproduktion bei. Und gerade hier können die Aktionen eines jeden Einzelnen eine Rolle spielen. Mit jeder Entscheidung anstatt eines tierischen ein rein pflanzliches Produkt zu kaufen und zu essen, kann ein Einfluss auf den fortschreitenden Klimawandel genommen werden (Poore \& Nemecek, 2018, S. 5).  Auch wenn die Hilflosigkeit über den geringen eigenen Einfluss auf große Probleme erdrückend ist, basiert die zukünftige Welt und Umwelt auf den eigenen zu jedem Zeitpunkt getroffenen Entscheidungen.
