\section{HA 07: fiktives Forschungsarbeit}
\subsection{Einleitung:}
Für körperlich eingeschränkte Menschen besteht seit neuestem die Möglichkeit, ihre fehlenden Funktionen durch in das Gehirn eingepflanzte Chips zu ersetzen und wiedererlangen zu können. Dabei werden die elektrischen Impulse im Gehirn gemessen und visualisiert. Durch die Variation der Impulse wird auf einem Monitor der Gedankengang angezeigt.% (zu unspezifisch)
Das wirft die Frage auf, inwieweit die Technologie für die Bedienung eines Computer/Smartphones verwendet werden kann.
Dies würde die Notwendigkeit eines Touchscreens beenden. Nicht nur die praktischen Vorteile des Wegfalls eines klobigen Gegenstands überwiegen, sondern auch die Übertragungsgeschwindigkeit zwischen Mensch und Maschine werden drastisch erhöht (Musk, 2017). %(zu umgangssprachlich)
Ob diese Möglichkeit mit dem heutigen Stand der Technik erreicht werden kann, wird in dieser Arbeit untersucht. Dabei wird auf die Hilfsmittel eingegangen, welche aktuell schon genutzt werden und mit den verfügbaren Chips und den Studien zu diesen verglichen. Diverse Studien bestehen bisher zu einer Anwendung im Labor. Diese werden eingehend in Bezug auf gesundheitliche Sicherheit, sowie die Umständlichkeit der Anwendung untersucht.(Untersuchungsgegenstände etwas genauer in Bezug auf die Forschungsfrage benennen
Die interdisziplinär auftretenden Probleme dabei sind gleichzeitig in der Medizin, als auch der Informatik und Elektrotechnik verankert, was den umfangreichen Theorieteil nötig macht. Danach werden die analysierten Studien vorgestellt und im Anschluss auf die zu untersuchenden Punkte geprüft.
%(Fragestellung fehlt)
%Forschungsfrage:
Sind eingepflanzte Chips alltagstauglich bzw. können sie die Bedienung von Smartphones ersetzen?
% Wird die Forschungsfrage aufgegriffen?
% Passen Einleitung und Fazit zusammen?
% Zusammenfassung der wesentlichen Erkenntnisse?
% Trichterprinzip?
\subsection{Fazit:}
In der vorliegenden Arbeit konnte anhand der aktuellen Studienlage gezeigt werden, dass die Bedienung eines Smartphones prinzipiell möglich ist. Dies wurde schon in Pilotprojekten gezeigt. Eine Bedienung der gewohnten Funktionen wie wir sie kennen und ohne Probleme die im Alltag benötigten Anwendungen zu nutzen ist aber nocht nicht erreicht worden.
Die Bedienung über kontrollierte Gedankenmuster konnte die Gesten auf einem Smartphone steuern. Diese Gesten haben ausgereicht, um Apps zu öffnen und voreingestellte Funktionen auszuführen (z.B. einen Anruf tätigen). Dies hat für bestimmte Fälle Vorteile gebracht, für eine alltägliche Nutzung waren diese Funktionen aber nicht weitreichend und detailliert genug.
Folgende Entwicklungen sind dazu noch nötig. Die Erkennung von Text und das "Diktieren" im Kopf muss sicherer und verlässlicher werden. Die Oberfläche für Smartphones, bzw das Bedienumfeld eines Chips im Kopf muss neu gedacht und pragmatisch neu festgelegt werden. Erst wenn feststeht, wozu Chips im Kopf genutzt werden sollen und die Chancen erkannt wurden, kann an der Entwicklung einer passenden Schnittstelle gearbeitet werden.
%(Zusammenfassung etwas kurz geraten, evtl noch mehr auf die Studienergebnisse eingehen. Außerdem fehlt die Bewertung der eigenen Arbeit. Ausblick und abschließender Satz gut)
