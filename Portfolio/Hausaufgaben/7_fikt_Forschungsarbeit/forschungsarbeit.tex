\section{HA 07: fiktives Forschungsthema - implantierte Mensch-Maschine Schnittstelle}
\subsection{Einleitung:}
  Während sich viele Menschen nicht vorstellen können, elektrische Geräte in ihren Körper einzupflanzen, können andere nicht mehr ohne. Um die Gehirnaktivität zu messen und so fehlende Funktionen wie die Steuerung von Prothesen zu ermöglichen, werden im Gehirn eingepflanzte Chips schon verlässlich in der Praxis eingesetzt. Diese können mit der Peripherie kommunizieren und einfache Gedankengänge aufzeichnen.\\
  Das wirft die Frage auf, inwieweit die Technologie für die Bedienung eines Endgerätes verwendet werden kann. Dies würde die Notwendigkeit eines Touchscreens bei Smartphones beenden, womit das klobige Gerät langfristig wegfallen würde. Auch die Übertragungsgeschwindigkeit zwischen Mensch und Maschine würde drastisch erhöht werden (Musk, 2017). \\
  Ob diese Möglichkeit mit dem heutigen Stand der Technik erreicht werden kann, wird in dieser Arbeit untersucht. Dabei wird auf die Hilfsmittel eingegangen, welche aktuell schon genutzt werden und mit den verfügbaren Chips und den Studien zu diesen verglichen. Diverse Studien bestehen bisher zu einer Anwendung im Labor. Diese werden eingehend in Bezug auf gesundheitliche Sicherheit, Anwenderfreundlichkeit und Verlässlichkeit untersucht.\\
  Die konkrete Forschungsfrage lautet demnach:
  %Forschungsfrage:
  Sind eingepflanzte Chips alltagstauglich und können sie die Bedienung von Smartphones ersetzen?\\
  Die interdisziplinär auftretenden Probleme sind gleichzeitig in der Medizin, als auch der Informatik und Elektrotechnik verankert, was den umfangreichen Theorieteil nötig macht. Danach werden die analysierten Studien vorgestellt und im Anschluss auf die zu untersuchenden Punkte geprüft.
% Wird die Forschungsfrage aufgegriffen?
% Passen Einleitung und Fazit zusammen?
% Zusammenfassung der wesentlichen Erkenntnisse?
% Trichterprinzip?
\subsection{Fazit:}
  In der vorliegenden Arbeit konnte anhand der aktuellen Studienlage gezeigt werden, dass die Bedienung eines Smartphones mit Hilfe von im Körper eingepflanzten Chips prinzipiell möglich ist. Dies wurde in Pilotprojekten gezeigt. Eine Bedienung der gewohnten Funktionen wie wir sie kennen und ohne Probleme die im Alltag benötigten Anwendungen zu nutzen ist jedoch bisher nicht erreicht worden.\\
  Die Bedienung über kontrollierte Gedankenmuster konnte die Gesten auf einem Smartphone steuern. Diese Gesten haben ausgereicht, um Apps zu öffnen und voreingestellte Funktionen auszuführen (z.B. einen Anruf zu tätigen). Dies hat für bestimmte Fälle Vorteile gebracht, für eine alltägliche Nutzung waren diese Funktionen aber nicht weitreichend und detailliert genug.\\
  Die drei Forschungsgegenstände gesundheitliche Sicherheit, Anwenderfreundlichkeit und Verlässlichkeit wurden bei  durchgeführten Studien bewertet und fallen unterschiedlich stark aus. Es lässt sich keine allgemeingültige Bewertung ableiten, da größtenteils unterschiedliche technische Ansätze verfolgt wurden. Um die Forschungsfrage genauer zu beleuchten wären demnach die Betrachtung von einzelnen viel versprechenden technischen Ansätzen nötig, was in dieser Arbeit nicht gemacht wurde.\\
  Es lassen sich jedoch technische Teilbereiche ableiten, welche für die Umsetzung des Ziels nötig sind. Die Erkennung von Text und das "Diktieren" im Kopf muss sicherer und verlässlicher werden. Die Oberfläche für Smartphones, bzw das Bedienumfeld eines Chips im Kopf muss neu gedacht und pragmatisch neu festgelegt werden. Erst wenn feststeht, wozu Chips im Kopf genutzt werden sollen und die Chancen erkannt wurden, kann an der Entwicklung einer passenden Schnittstelle gearbeitet werden.
%(Zusammenfassung etwas kurz geraten, evtl noch mehr auf die Studienergebnisse eingehen. Außerdem fehlt die Bewertung der eigenen Arbeit. Ausblick und abschließender Satz gut)
