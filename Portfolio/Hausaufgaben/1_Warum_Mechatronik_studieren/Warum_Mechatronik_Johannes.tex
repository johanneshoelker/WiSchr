\chapter{Hausaufgaben}
\section{HA 01: Warum sollte man Mechatronik studieren?}
  Computer, Auto, Smartphone, nahezu jedes Gerät aus deinem Alltag funktioniert so wie du es willst... Aber wie ist das möglich? Diese technischen Zusammenhänge zu erkennen und zu verstehen bringt der Studiengang Mechatronik mit sich.\\
  Im Grundstudium lernt man die Grundlagen aus den technischen Teilbereichen Maschinenbau, Elektrotechnik und Informatik kennen. Anschließend wird auf die Symbiose dieser Bereiche eingegangen und am Ende kann man sich nach den eigenen Interessen richten und sich spezialisieren. So werden dir im Laufe des Studiums die Abläufe vieler technischer Geräte und Prozesse klar.\\
  Dieses breit gefächerte Wissen ist nach abgeschlossenem Studium in deinem Repertoire und man hat anschließend die freie Wahl, auf welchen Fachbereich man sich nach dem Mechatronikstudium spezialisieren möchte oder ob man Schnittstellen der Fachbereiche koordiniert.\\
  Lass dir nach deinem Abitur nicht die Möglichkeiten nehmen und fokussiere   dich nicht zu früh auf einen Fachbereich, denn wer weiß welche Interessen du nach dem Studium haben wirst. Der Bachelor in Mechatronik qualifiziert dich für den Master in nahezu jedem technischen Zweig, sowie natürlich für einen Master in Mechatronik.\\
  Wenn deine Stärken in der Mathematik oder der Physik liegen bist du bei Mechatronik genau richtig. Aber auch wenn du nicht viel Vorerfahrung hast, kann der Studiengang Mechatronik was für dich sein. Du lernst die Grundlagen neu kennen und sie anschließend anwenden. Die Themen werden sehr grundsätzlich erklärt und mit dem nötigen Interesse kann man auch fachfremd in das Studium einsteigen.
