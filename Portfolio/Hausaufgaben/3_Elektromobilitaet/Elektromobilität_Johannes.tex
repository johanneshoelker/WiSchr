\chapter{HA 03: Elektromobilität für Grundschulkinder}
Autos ohne Lärm und ohne stinkende Abgase? Das könnte bald schon wahr werden. Bisher hört ihr Autos schon von weitem und könnt so wissen, dass ihr nicht auf die Straße gehen dürft. Das wird sich in nächster Zeit ändern. Ein Elektroauto funktioniert nämlich anders als die bisherigen Autos und ist dabei komplett lautlos. Deshalb ist es jetzt noch wichtiger: Verlasst euch nicht auf eure Ohren, sondern schaut zur Sicherheit nochmal nach links und rechts bevor ihr über die Straße lauft!\\
Ein Elektroauto ist sehr einfach aufgebaut. Es gibt einen Elektromotor und einen Stromspeicher. Der Elektromotor macht aus elektrischem Strom eine drehende Bewegung. Diese Drehbewegung wird an die Reifen weitergegeben und das Auto kann fahren. Der elektrische Strom kommt dabei aus dem Stromspeicher, also dem Akku. Der Akku speichert den elektrischen Strom und muss an der Tankstelle aufgeladen werden. Das dauert aber länger als früher, weshalb bei langen Autofahrten bald auch eine lange Pause gemacht werden muss. Dann können deine Eltern nicht mehr "Nein" sagen, wenn ihr was essen wollt!\\
Aber nicht nur Autos werden leise und ohne Abgase sein. Auch alle anderen Fortbewegungsmittel können mit Strom betrieben werden. E-Scooter zum Beispiel fahren auch mit einem Elektromotor, genauso wie E-Fahrräder und auch Züge fahren damit. Wenn ihr groß seid, werdet ihr womöglich nur noch mit Elektromotoren fahren und das macht besonders Spaß, denn sie beschleunigen sehr schnell. Das Getriebe wie bei vorigen Autos wird nicht mehr benötigt.\\
Elektromobilität macht also sehr viel Spaß aber wird dabei auch noch gefährlicher. Also wie schon erwähnt, bleibt weiter vorsichtig im Straßenverkehr und passt auf euch auf!
