\section{HA 06: Welche Forschungsprojekte sollten an Universitäten in Zukunft stärker gefördert werden?}
% Einleitung:
  Naturbasierte Lösungen (NBL): So lautet einer der bei der COP in Glasgow genannten Ansatzpunkte, um den anhaltenden Klimawandel noch aufzuhalten. Die zentrale Forderung lautet dabei, die Natur in ihrer ursprünglichen Form zu schützen und schon vom Menschen beeinflusste Gebiete zu renaturieren \cite{wahnbaeck_nature-based_2021}. Es gibt verschiedene Wege NBL umzusetzen.\\\\
% Ansatzpunkt Uni:
  Universitäten bieten den Ansatz, Lösungen zu finden, sie weiterzuentwickeln und für die breite Masse anwendbar zu machen. Anschließend können Empfehlungen formuliert werden, welche von Politik und Unternehmen übernommen werden sollten. Die von der Wissenschaft erarbeiteten Lösungen sind dabei frei von subjektiven Meinungen und bieten eine Grundlage, auf derer die Politik Entscheidungen treffen kann.\\
% Forschung
  Um die passenden Empfehlungen für einen funktionierenden Klimaschutz zu bekommen, sollte dabei im Vorfeld in die richtige Richtung geforscht werden.\\
% Zitation Bund:
  Das Bundesministerium für wirtschaftliche Zusammenarbeit und Entwicklung (BMZ) hat bereits eine klare Haltung zu NBL und fördert proaktiv deren Einsatz. "Das in Wert setzen der Natur kann eine Schlüsselrolle spielen, um Klimaanpassung zu fördern und Katastrophenrisiken zu verringern." \cite{bundesministerium_fur_wirtschaftliche_zusammenarbeit_und_entwicklung_naturbasierte_2021} Aus diesem Grund ist nun die Wissenschaft gefragt.\\\\
% urbaner Wald
  Eine vielversprechende NBL ist der urbane Wald, bzw. die allgemeine Begrünung von Städten. Diverse Vorteile ergeben sich dadurch: "Der Wald filtert die Luft, nimmt Regenwasser auf und kühlt die sich aufheizende Stadt in Zeiten der Klimakrise. Die Anwohner können Natur vor ihrer Haustür erleben. Und die Stadt spart Geld, das sie sonst in pflegeintensive Parks stecken müsste." \cite{wahnbaeck_nature-based_2021}.\\\\
  Ein aktueller Leitfaden der Universität Osnabrück liefert ein gutes Beispiel für die Formulierung von Empfehlungen. Dieser gibt Handlungsempfehlungen für Dachbegrünung in Innenstädten, welche direkt umsetzbar sind und der Stadtverwaltung, sowie Privatleuten helfen, ihre Dächer zu bepflanzen \cite{schroder_extensive_2020}.\\
% Stadtbegrünung
  Diesem Vorbild folgend können tiefergehende Entwicklungen für die Begrünung von Städten entstehen, zum Beispiel technische Lösungen für die Bepflanzung von Schrägdächern, welche bei der Sanierung von Dachziegeln angewandt werden können. Da viele Dinge, wie Bewässerung und Pflege, sowie der Schutz der Bausubstanz beachtet werden müssen, ist hier Forschungspotential gegeben.\\\\
% Renaturierung:
  Nicht nur der urbane Raum bietet Möglichkeiten der nachhaltigen Umgestaltung. Auch auf dem Land gibt es Potential für NBL. Die Renaturierung von Flussauen oder die Nutzungsänderung von landwirtschaftlich genutzten Flächen zu Forstflächen bieten ein großes Forschungsfeld. Die Erfassung von Tierarten und der Einfluss auf die Erholung der Bestände kann in langfristigen Studien erfasst werden.\\
% weitere Forschungspotentiale
  Gleichzeitig haben Universitäten die Möglichkeit, die Bevölkerung aufzuklären und Fakten zu schaffen. Aus einer Studie der Hochschule Bingen geht zum Beispiel hervor, dass Gründächer gegenüber Kiesdächern bezüglich Biodiversität, Mikroklima und Wasserhaushalt bedeutend wertvoller sind \cite{hietel_extensive_2016}.
