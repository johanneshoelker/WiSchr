\chapter{HA 02: Welche Technologie wird in den nächsten 5 Jahren von immenser Bedeutung sein?}
%„Überleitung von der Einleitung verbessern. Hier habe ich das gefühl das ich plötzlich reingeworfen werde “ zb Unsere welt wird häufiger von umweltkatasprophen heimgesucht , was auf die Erderwärmung zuruck zu führen ist....
Um die Lebensgrundlagen der Menschheit und aller weiteren Tierarten nicht weiter zu gefährden und zu zerstören, darf die globale Erderwärmung keine 1,5° Celsius überschreiten \cite{masson-delmotte_ipcc_2019}.\\
Um die schon jetzt auftretenden Folgen des anhaltenden Klimawandels noch stoppen zu können, kann an vielen unterschiedlichen Stellen etwas getan werden. Dazu gehört unter anderem der Verkehrssektor, der Ernährungssektor und vor allem der Energiesektor. Weltweit ist der Energiesektor der größte Emittent von Treibhausgasen, die Verbrennung fossiler Brennstoffe beträgt 88\% der CO2-Emmissionen, und das muss sich zwangsläufig und innerhalb kurzer Zeit ändern \cite{quaschning_regenerative_2019}.\\
Die nötigen Mittel das zu erreichen haben die Menschen schon erfunden. Mit Solar- und Winkraftanlagen gibt es zwei ausgereifte Technologien, welche saubere Energie liefern. Diese Energieproduzenten können in Deutschland, sowie auf der ganzen Welt errichtet werden. \\
Jedes Hausdach kann potenziell ein Solarkraftwerk werden. Deshalb sollte bei Neubauten auf eine Abdeckung mit Solarplatten geachtet werden. Die Technologie ist insofern ausgereift, dass die aktuell zum Verkauf stehenden Solarplatten technisch nicht mehr verbessert werden können. Lediglich eine thermische Nutzung mit Wasser kann den Wirkungsgrad noch erhöhen. Das erwärmte Wasser kann dann für die Heizung oder das Warmwasser genutzt werden.\\
Noch nicht ausgereift ist die Energiespeicherung. Diese ist essentiell für eine erfolgreiche Energiewende. Das bedeutet, dass unsere Nutzung von Energie sich zwangsläufig an die Energiegenerierung anpassen muss. Nur wenn die Sonne scheint und der Wind weht, wird Energie generiert und bei einem Überschuss werden die Akkus geladen. Wenn diese irgendwann leer sein sollten, kann auch keine Energie mehr bezogen werden.\\
Ein großer Vorteil der Technologie Solar ist die Möglichkeit der dezentralen Verteilung. Mit einem weit verbeiteten Netz aus vielen Akkus, welche sich dann gegenseitig unterstützen, kann trotzdem eine durchgehende Versorgung gewährleistet werden. Erst durch die polyzentral verteilte Generierung und Speicherung von Solarstrom, welche durch Batteriestationen in jedem Haus erreicht wird, können wir unabhängig von fossilen Energieträgern werden.\\
%"Unabhängig werden" ist dabei ein zentraler Begriff. Ohne die Verlegung von langen Leitungen kann ein Haushalt in der Peripherie mit einer sogenannten Inselnetzanlage autark mit Strom versorgt werden.
%An das Netz gekoppelte Systeme dagegen bekommen den überschüssigen Strom ausbezahlt, das macht einen Hausbesitzer/ eine Hausbesitzerin mit großer Dachfläche zu einem Energielieferanten und im besten Fall kann man davon sogar leben.
