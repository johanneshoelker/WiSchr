\section{HA 04: Erneuerbare Energien} \label{ha04}
% Drastische Einleitung
  Das 1,5-Grad Ziel ist aktuell nur noch schwer zu erreichen. "Eine globale Erwärmung von 1.5\textcelsius\ und 2\textcelsius\ wird im Laufe des 21. Jahrhunderts überschritten werden, es sei denn, es erfolgen in den kommenden Jahrzehnten drastische Reduktionen der CO2- und anderer Treibhausgasemissionen." \cite{ipcc_climate_2021}. Die entstehenden Extremwetterereignisse wirken sich auf die gesamte Erde aus. Um das zu verhindern ist es für jeden Menschen relevant, etwas dafür zu unternehmen die Erderwärmung zu begrenzen.\\
% Treibhausgasemittenten:
  Der größte Emittent von Treibhausgasen ist der Energiesektor mit einem Anteil von 73,2\%, gefolgt von der Nahrunsgmittelproduktion mit 18,4\% \cite{ritchie_co_2020}. Deshalb ist es immens wichtig im grösten Sektor der Treibhausgasemmissionen nach klimafreundlicheren Lösungen zu suchen, was bedeutet erneuerbare Energien zu fördern.\\
% Energiesektor:
  Dass der Energiesektor einen so hohen Anteil an den Treibhausgasemmissionen hat liegt neben dem enorm hohen Konsum von Energie in der heutigen Zeit daran, dass 2020 85\% des weltweiten Primärenergiebedarfs klimaschädliche Energieträger lieferten \cite{quaschning_regenerative_2019}. Die Verbrennung von Kohle, Erdöl und Erdgas trägt demnach maßgeblich zur globalen Klimakatastrophe bei und das muss geändert werden.\\
% Lösungsansatz erneuerbare
  Ändern kann man dies urch den Einsatz von erneuerbaren Energien. Die Sonne liefert jeden Tag soviel Energie, dass allein die Fläche der Sahara ausreichen würde die 200-fache weltweit benötigte Energie zu liefern. Die Nutzung der Energie, welche auf die Fläche Niedersachsens trifft, würde demnach den weltweiten Energiebedarf decken \cite{quaschning_regenerative_2019}.\\
% wie nutzt man sonnenenergie?
  Die direkte Nutzung von Sonnenstrahung ist mithilfe von Photovoltaik möglich. Solarzellen, welche mithilfe des Photoeffekts sehr hohe Wirkungsgrade erzielen, können immer dann Strom erzeugen, wenn es hell ist. Indirekt wird die Sonnenenergie über Wind oder Wasserkraft genutzt. Wie bei Photovoltaik kann auch hier nur dann Strom erzeugt werden, wenn der Wind weht oder genug Wasser zum Antribe von Turbinen vorhanden ist. Die Entwicklung dieser Technologien ist weit fortgeschritten, sodass sie im großen Stil eingesetzt und lange betrieben werden können \cite{quaschning_regenerative_2019}.\\
% Energiespeicher:
  Scheint keine Sonne oder weht kein Wind, stoßen  erneuerbare Energien jedoch auf ein zentrales Problem, denn dann liefern sie keinen Strom mehr. Um die Versorgungslücke zu schließen, muss demnach Energie als Puffer zwischengespeichert werden. Die Entwicklung von Energiespeicherlösungen ist dabei noch nicht ausgereift und die Herstellung handelsüblicher Batterien erfordert den Einsatz von Ressourcen, die nur schwer zu erreichen sind.\\
% Fokus der Forschung:
  Demnach sollte im Fokus der Forschung die Entwicklung neuer nachhaltiger Energiespeicher stehen. Des Weiteren ist der schnelle Aufbau von erneuerbaren Energiegeneratoren wichtig. Die Frage ist nicht, wie können wir die Technologien verbessern, sondern wie können wir so viele von den bestehenden Technolgien so schnell wie möglich in der Praxis einsetzen und aufbauen.\\
% Lösungsansätze:
  Um den Einsatz von Menschen zu verringern und die Effizienz von arbeitenden Elektroinstallateuren und Ingenieuren zu erhöhen, kommen Entwicklungen in der Automatisierung und Planung durch KI-Systeme in Frage. Aber auch politische Entscheidungen und der Wille in der Bevölkerung tragen zur Energiewende bei. Erst wenn alle Akteure an einem Strang ziehen, kann die Klimakatastrophe vermieden werden.
