\section{HA 05: Artikelrezension}
% Das wird behandelt:
  In dem Artikel von \cite{jiang_development_2017} "Development of a Motion Controlled Robotic Arm" wird ein technisches System vorgestellt, in welchem ein Roboterarm mit der menschlichen Armbewegung gesteuert wird. Dabei steht die technische Umsetzung im Vordergrund.\\
% Verortung im Forschungsdiskurs:
  Eine klare Begründung für den Bau und den Einsatz eines solchen Systems wird in der Einleitung mit der Aussage des Einsatzes in gefährlichen Bereichen geliefert. Eine wissenschaftliche Recherche, inwieweit das entwickelte System schon von anderen gebaut wurde oder diverse Vorgänger besitzt, wurde dabei nicht durchgeführt.\\
% Gelungen:
  Die Hardware, bestehend aus dem Arduino Uno, 4 Servomotoren und diversen Sensoren wird eingehend erklärt und die Funktionsweise ist klar verständlich. Auch die Zusammensetzung der Komponenten ist ersichtlich. Zum Beispiel kann man gut nachvollziehen, wie die Stromversorgung ausgelegt wurde.\\
% Unverständlich/ Fehlerhaft/ Unverständlich und offene Fragen:
  Die Beschreibung der Software ist dagegen wenig ausführlich. Es wird erwähnt welche Ein- und Ausgaben der Mikrocontroller hat und wie die Positionierung des Roboters mithilfe des Kompasses abläuft. Dabei bleibt aber unklar, wie die Berechnung der einzelnen Winkel für die Servomotoren aussieht und wie sie programmatisch umgesetzt wurde. Dies wird nur mit einem Satz erwähnt: "Additionally, two of the accelerometer outputs are used to determine the vertical angles that the bicep and the forearm of the robot should be at." Diese fehlende Information ist wichtig, da sie den Kern der Arbeit ausmacht und das größte zu lösende Problem darstellt.\\
  Des Weiteren wird nicht auf die Theorie eingegangen, welche hinter der Kinematik eines menschlichen Arms steht. Das wäre bezüglich der Auswahl der Servomotoren relevant gewesen und bleibt ungeklärt.
%TODO:
