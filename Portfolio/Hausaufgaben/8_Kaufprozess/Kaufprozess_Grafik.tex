\section{HA 08: Beschreibung Kaufprozess Elektronikkauf}
  Bevor neue Elektronikgeräte gekauft werden, sollte man sich besonders lange mit der Frage beschäftigen: "Brauche ich das wirklich?" Der Aufwand an Ressourcen und Energie steht oft in keinem Verhältnis zu der Anwendung. Oft stehen die Geräte nach zweimaliger Benutzung im Schrank und dienen nur als Deko. Erst wenn man das Gerät wirklich akut braucht, sollte man meiner Meinung nach über einen Kauf nachdenken.\\\\
  Ist die Situation eingetreten, bei der die gewünscht Elektronik wirklich benötigt wird, lässt sie sich in vielen Fällen leihen oder mieten. Eine weitere Alternative ist der Gebrauchtkauf, bei der weniger Geld ausgegeben und nachhaltiger gehandelt wird.\\\\
  Tritt der Fall öfter ein, dass das Gerät gebracuht wird und Mieten und Gebrauchtkauf kommen nicht in Frage, kann man sich über das Angebot der unterschiedlichen Anbieter eingehend informieren um einen Fehlkauf zu vermeiden. Ist dann teils aus dem Bauchgefühl und teils auf einer subjektiven Entscheidung heraus das passende Gerät gefunden, kann noch nach Vergünstigungnen suchen und in unterschiedlichen Läden, online wie offline, suchen um das perfekte Angebot zu finden.
