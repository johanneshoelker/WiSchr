\subsection{Gliederung}
0 Abstract\\
1 Einleitung\\
  1.1 Motivation\\
  1.2 Zielstellung\\
  1.3 Aufbau der Arbeit\\
2 Theorie\\
  2.1 Arbeitsplatzgestaltung\\
    2.1.1 Gestaltungsrichtlinien\\
    2.1.2 Telepräsenz\\
      Immersion\\
      Intuition\\
    2.1.3 Anthropometrie\\
      menschlicher Sehraum\\
      Oberkörperproportionen\\
  2.2 visuelle Kommunikation\\
    2.2.1 Aufnahme - Fish-Eye Kamera\\
    2.2.2 Vorverarbeitung - radiale Verzerrung\\
    2.2.3 Darstellung - Head Mounted Display\\
  2.3 Robotik\\
    2.3.1 Roboterarten\\
    2.3.2 Manipulatorsteuerung\\
    2.3.3 anthropomorphe Roboterkinematik\\
    2.3.4 Server/Datenübertragung\\
    2.3.5 Stand der Technik/bisherige Lösungsansätze\\
  2.4 Forschungsfrage\\
3 Methoden\\
  3.1 Motion Tracking\\
  3.2 Manipulator des Instituts\\
  3.3 Bildübertragung\\
    3.3.1 HTC Vive\\
    3.3.2 SteamVR\\
    3.3.3 StereoStitch Live\\
    3.3.4 Kodak Pixpro SP360 4K\\
  3.4 Robot Operating System\\
4 Konzept\\
5 Umsetzung\\
  5.1 Kontrolle des Roboterarms\\
    5.1.1 udplistener.py\\
    5.1.2 Angles.msg\\
    5.1.3 roscore, rosserial\_arduino\\
    5.1.4 Arduino\\
  5.2 Bildübetragung\\
  5.3 Hardwaresetup\\
    5.3.1 Kameraposition\\
6 Funktionsvalidierung\\
7 Diskussion und Ausblick\\
8 Literaturverzeichnis
