\subsection{Textprobe: vorläufige Einleitung}
%Ist der Text verständlich?
%Einhaltung sprachlicher und formaler Standards
  Zu Beginn dieser Arbeit wird die Motivation erläutert, aus welcher diese heraus geschrieben wird. Anschließend leitet sich daraus die Zielstellung ab und ein Überblick über die folgenden Kapitel wird präsentiert.
\subsubsection{Motivation}%Warum möchte ich das erforschen?
  Der erste ferngesteuerte Roboter, 1949 erfunden von Raymond Goertz, wurde dazu eingesetzt radioaktives Material gefahrenfrei zu bewegen. Die eingesetzte Methode des Master-Slave Manipulators übersetzte mechanisch die Bewegung des menschlichen Arms auf einen Manipulator. So konnten Gefahrengüter hinter einer durchsichtigen Trennwand bewegt werden. Dabei merkte Goertz an, dass die anwendende Person aus einem anderen Blickwinkel auf die auszuführende Aufgabe schauen muss und die Aufgabe so erschwert wird \cite{goertz_master-slave_1949}. \\\\
  Diese Problematik ist in der Robotik größtenteils durch eine Automatisierung der Manipulatoren gelöst worden. Durch ein hohes Maß an Sensorik und in kürzester Vergangenheit auch durch den Einsatz neuronaler Netze können sich Roboter weitgehend selbstständig bewegen \cite{siciliano_springer_2008}.\\\\
  % Ausführung von für den Menschen gemachten Arbeitsaufgaben
  Es gibt jedoch weiterhin viele Aufgaben, bei denen ein Mensch anwesend sein muss. Die Wartung von industriellen Maschinen, Pflegetätigkeiten und kooperatives Arbeiten sind nur ein paar Beispiele. Viele Arbeitsumgebungen sind des Weiteren für den Menschen konzipiert und die Aufgaben sind komplex und damit nicht direkt automatisierbar \cite{fritsche_first-person_2015}. Um dennoch die Aufgaben lösen zu können, ohne dass ein Mensch anwesend ist, kann ein anthropomorpher ferngesteuerter Manipulator eingesetzt werden \cite{tanie_mfi_2003}\cite{stanczyk_development_2006}.\\\\
  Herkömmliche Programmierverfahren erfordern des Weiteren viel Übung und Zeitaufwand. Die programmierende Person muss eine spezielle Ausbildung haben und braucht dennoch gewisse Zeit für die Umsetzung. Besonders in kleinen und mittelständischen Unternehmen bringen günstige Robotersysteme, welche einfach zu installieren, einzustellen, zu programmieren und zu warten sind, einen Vorteil. Um die genannten Arbeitsschritte zu vereinfachen erfordert es eine intuitive Steuerung des Roboters \cite[S.76]{brogardh_present_2007} \cite[S.190]{ehlers_echtzeitfahige_2019}. \\\\
  Räumlich in die Rolle des Roboters zu schlüpfen und aus der veränderten Perspektive heraus zu handeln eignet sich für eine Manipulatorsteuerung in für Menschen gefährlichen Umgebungen \cite[S.9]{mareczek_grundlagen_2020}. Diverse Konzepte und Systeme um diese Problematik zu lösen wurden entwickelt und werden in Kapitel 2 vorgestellt.\\\\
  % Mensch vor Technik:
  In der vorliegenden Arbeit wird demnach ein Arbeitsplatz gestaltet, welcher den Menschen mit Technik verbindet. Es werden Geräte eingesetzt, welche der Mensch tragen soll und so sein Umfeld beeinflussen. Für die gesamte Arbeitsgestaltung ist dabei wesentlich, dass nicht die technische Machbarkeit, sondern die "menschlichen Bedarfe und Möglichkeiten" im Vordergrund stehen \cite{strater_positionspapier_2019}. Für alle Entwicklungsschritte wird diese Richtlinie bedacht und evaluiert.\\\\
  % OpenSource für Nachhaltigkeit
  Aufgrund der aktuellen Situation des globalen Klimas \cite{ipcc_climate_2021}, soll die technische Entwicklung auch in Hinblick auf Nachhaltigkeit geschehen. Der Einsatz von Open Source Design (OSD) kann nach \cite{setchi_implications_2016} während und zum Ende des Lebenszyklus eines technischen Produkts einen positiven Einfluss auf Nachhaltigkeit haben. Die Reparierbarkeit, Wiederverwendbarkeit und die lokal-gebundene Wertschöpfung werden durch Ntzung von OSD ermöglicht und haben damit eine erhöhte Effizienz bei der Ressourcennutzung.
\subsubsection{Zielstellung} \label{1:sec:ziel}
  An die bereits bestehenden Konzepte soll diese Arbeit anknüpfen und die Entwicklung eines immersiven Systems zur Manipulatorsteuerung wird präsentiert. Die nutzende Person soll möglichst intuitiv einen Manipulator steuern können, ohne dass Vorerfahrung vorhanden ist. Dabei lautet das zentrale Ziel, die nutzende Person visuell in die Rolle des Roboters schlüpfen zu lassen. \\\\
  Um dies zu erreichen wird mit den Technologien Motion-Capturing und Head-Mounted Displays (HMD) gearbeitet. Bei der Entwicklung des Manipulators wird soweit möglich mit OSD gearbeitet, in diesem Fall bedeutet dies den Einsatz von 3D-Druck und die Nutzung des Robot Operating Systems (ROS). Bei der gesamten Entwicklung soll des Weiteren ein auf den Menschen zugeschnittenes System entwicklet werden, wobei nicht die technische Machbarkeit im Vordergrund steht.
\subsubsection{Aufbau der Arbeit}
  Nachdem nun die Hintergründe für die Entstehung dieser Arbeit erläutert wurden, werden zunächst die für das Verstehen der Arbeit nötigen wissenschaftlichen Erkenntnisse in Kapitel 2 zusammengefasst. Der Begriff und die Erreichung von Telepräsenz wird in Kapitel 2 erläutert, sowie die Grundlagen für Manipulatoren in Kapitel 2 gegeben\\
  In Kapitel Kapitel 3 werden die eingesetzten Methoden erklärt. Die der Umsetzung zugrundeliegende Konzeptidee wird in Kapitel 4 präsentiert und erläutert.\\
  Anschließend wird auf das Vorgehen bei der Realisierung des Konzepts in Kapitel 5 eingangen und aufgetretene Problematiken werden benannt.\\
  Das entstandene System wird in Kapitel 6 auf die Funktion hin geprüft und validiert. Die benannten Problematiken und Schwierigkeiten bei der Umsetzung werden in Kapitel 7 diskutiert und ein Ausblick auf die zukünftige Forschung wird gegeben.\\
