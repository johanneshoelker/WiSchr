\subsection{Fragestellung}
%Wird einem potenziellen Betreuer ersichtlich, woran ich forschen möchte?
%Sind meine Ausführungen überzeugend und helfen sie, mein Thema zu "verkaufen"?
% Problemstellung:
Die Programmierung und Steuerung von Manipulatoren geschieht zur Zeit über ein Umdenken der anwendenden Person indem sie von einer peripheren Perspektive auf das System schaut. Das kann Komplikationen erzeugen, erfordert Einarbeitung und Erfahrung und macht die Steuerung schwierig \cite[S.190]{ehlers_echtzeitfahige_2019}. Die Programmierung aus der Ferne ist dementsprechend aufwendig, da sie mit Hilfe von Kameras aus verschiedenen Blickwinkeln erfolgen muss.\\\\
Indem die anwendende Person ihren Arm bewegen und einen Manipulator so direkt steuern kann, werden bekannte Abläufe beibehalten. Dadurch kann nach \cite{blackler_towards_2007} Intuition erreicht werden. Für die direkte und unabhängige Steuerung ist wie in Kapitel 2 die anthropomorphe Gestaltung des Manipulators hilfreich. Die kinematischen Ketten von den in Kapitel 2 gezeigten Manipulatoren sind jedoch nicht direkt anthropomorph, da sie nicht alle Freiheitsgrade des menschlichen Arms besitzen.\\\\
% Zielsetzung:
Die Entwicklung eines anthropomorphen Manipulators, welcher die in Kapitel 2 genannten Freiheitsgrade besitzt, ist daher ein Ziel dieser Arbeit. Ein System, welches die Technologien der Robotik und Datenbrillen kombiniert, kann helfen die genannte Problematik des Umdenkens bei der Programmierung zu lösen. Durch den Einsatz von HMDs soll die anwendende Person die Perspektive des Roboters einnehmen und den Manipulator intuitiv steuern können.\\\\
% Forschungsfrage:
Die in dieser Arbeit behandelte Forschungsfrage lautet demnach: Wie lässt sich ein System konstruieren, welches immersive Telepräsenz bei der Steuerung eines anthropomorphen Manipulators erreicht und dabei ohne Vorwissen bedienbar ist?\\\\
%Vorgehen:
Um diese Frage zu beantworten wird zunächst auf die verwendete Methodik in Kapitel 3 eingegangen. Anschließend werden die unterschiedlichen Geräte und Programme in Kapitel 4 in Zusammenhang gebracht, sodass ein Konzept ensteht, welches in Kapitel 5 umgesetzt wird. Um die Beantwortung der Forschungsfrage messbar zu machen wird anschließend eine Funktionsvalidierung durchgeführt, welche das Kapitel Kapitel 6 beschreibt. Die abschließende Beantwortung und Diskussion der Forschungsfrage geschieht in Kapitel 7.
