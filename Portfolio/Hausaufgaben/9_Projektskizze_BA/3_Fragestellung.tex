\subsection{Fragestellung}
%Wird einem potenziellen Betreuer ersichtlich, woran ich forschen möchte?
%Sind meine Ausführungen überzeugend und helfen sie, mein Thema zu "verkaufen"?
% Problemstellung:
  Die Programmierung und Steuerung von Manipulatoren geschieht zur Zeit noch über ein Umdenken des Anwenders indem er von einer anderen Perspektive auf das System schaut und es so steuert. Das kann Komplikationen erzeugen, erfordert Einarbeitung und Erfahrung und macht die Steuerung schwierig \cite[S.190]{ehlers_echtzeitfahige_2019}. Die Programmierung aus der Ferne ist dementsprechend schwierig, da die Programmierung mithilfe von Kameras aus verschiedenen Blickwinkeln erfolgen muss.\\\\
% Zielsetzung:
  Ein System, welches die Technologien der Robotik und der Datenbrillen kombiniert, kann helfen diese Problematik zu lösen. Durch den Einsatz von HMDs soll die anwendende Person die Perspektive des Roboters einnehmen und den Roboterarm intuitiv steuern können. \\\\
% Forschungsfrage:
  Die in dieser Arbeit behandelte Forschungsfrage lautet demnach: Wie lässt sich ein System konstruieren, um immersive Telepräsenz zu erreichen und so eine intuitive Steuerung eines Manipulators zu ermöglichen?
%Alternative:
  %Im Fachgebiet A\&O der Uni Kassel wurde ein Roboterarm gebaut, welche durch die eigene Körperbeweung gesteuert wird, indem die Winkel des menschlichen Arms direkt auf die Winkel des Roboterarms gesendet werden. Die Freiheitsgrade des Roboterarms sind dementsprechend den Freiheitsgraden des menschlichen Arms nachempfunden.\\
  %Aufgrund der menschlichen Proportionen kam die Idee auf, inwieweit man den Arm als seinen eigenen annehmen kann, wenn an der richtigen Stelle eine Kamera montiert und auf eine VR-Brille gestreamt wird. \\
  %Dadurch würde sich eine inuitiv bedienbare Remote Steuerung eines Roboterarms ergeben, mit welchem Aufgaben, die vorher ein Mensch vor Ort ausgeführt hat, aus der Ferne übernommen und eingespeichert werden können. Besonders in Gefahrenbereichen ist dieses Konzept interessant.\\
