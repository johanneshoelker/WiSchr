\section{Fragestellung}
%Wird einem potenziellen Betreuer ersichtlich, woran ich forschen möchte?
%Sind meine Ausführungen überzeugend und helfen sie, mein Thema zu "verkaufen"?
%Problemstellung:
  Die Programmierung und Steuerung von Manipulatoren geschieht aktuell noch über ein Umdenken des Anwenders indem er von einer anderen Perspektive auf das System schaut und es so steuert. Das kann Komplikationen erzeugen, erfordert Einarbeitung und Erfahrung und macht die Werkzeugtransformation schwierig. Um die fortschreitende Automatisierung für eine breite Masse verfügbar zu machen, müssen Manipulatoren intuitiv bedienbar sein. \\
%Zielsetzung:
  Ein System, was die Technologien der Robotik und der Datenbrillen kombiniert, kann helfen diese Problematik zu lösen. Durch Einsatz von Telepräsenz soll die anwendende Person die Perspektive des Roboters einnehmen und den Roboterarm intuitiv steuern können. \\
%Forschungsfrage:
  Die in dieser Arbeit behandelte Forschungsfrage lautet demnach:
  Wie lässt sich ein System mit den beiden Technologien Robotik und Datenbrillen bauen, um funktionierende Telepräsenz zu erreichen und so eine intuitive Steuerung eines Manipulators zu ermöglichen?
